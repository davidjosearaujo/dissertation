\chapter{State of the Art}%
\label{chapter:State of the Art}

\begin{introduction}
This chapter will provide a comprehensive review of current \ac{5G} and Wi-Fi integration efforts, existing authentication mechanisms, and challenges in device identification. It will also explore recent developments and proposed solutions in the field, setting the context for our research.
\end{introduction}

\section{\ac{5G} Network Architecture}

\ac{5G} network represent a major shift from previous version in the sense that it is a \ac{SBA} which incorporates \ac{NFV} and \ac{SDN} technology. These changes allow for seperation between \ac{UP} and \ac{CP}, improving scalability, and flexibility of deployments, but most importantly, it allows for a unified authentication framework.~\cite{23.501-p56}

\subsection{Support for Non-3GPP}

\ac{3GPP} refers to the standards developed for mobile networks, including 3G, 4G, and \ac{5G}. These are cellular technologies used by mobile carriers to provide network services. Non-\ac{3GPP} refers to other access technologies not standardized by \ac{3GPP} but still capable of integrating with \ac{3GPP} networks, such as Wi-Fi or satellite networks. These networks can provide connectivity, but following a different standards (e.g., IEEE for Wi-Fi).

In \ac{3GPP} architecture, trusted and untrusted refer to the way a non-\ac{3GPP} network (such as Wi-Fi) connects to the mobile core network. A trusted access network is a network that has been verified and trusted by the mobile operator. It connects directly to the core network using secure protocols and behaves similarly to \ac{3GPP} networks. For example, a mobile operator’s managed Wi-Fi network might be treated as trusted, while something like a public Wi-Fi hotspot managed by another provider may be considered as untrusted as it does not meet the same security standards.

\subsection{How is it \ac{5G} different from 4G regarding security?}

Each generation of cellular networks has defined at least one new authentication method. For example, \ac{4G} introduced EPS-AKA, and 5G introduced three authentication methods: \ac{5G-AKA}, \ac{EAP-AKA’}, and \ac{EAP-TLS}.

From the point of view of authentication, a cellular network consists of three main components: \acp{UE}, a \ac{SN}, and a \ac{HN}.

\section{Wi-Fi Integration Challenges}

\section{Authentication and Identity Management in \ac{5G}}

\section{Current Solutions and Proposals}