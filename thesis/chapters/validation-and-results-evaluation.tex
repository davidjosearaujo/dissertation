\chapter{Validation and Results Evaluation}%
\label{chapter:validation-and-results-evaluation}

\begin{introduction}
This chapter presents the validation process and evaluates the performance and effectiveness of the implemented solution. Through a series of defined test scenarios and experiments, quantitative and qualitative results are gathered and analyzed. These findings are then critically assessed against the initial research objectives and compared with existing approaches discussed in the state of the art.
\end{introduction}

\section{Methodology}

% Define the overall approach for validating your solution (e.g., simulation-based testing, prototype testing).

%Specify the key performance indicators (KPIs) or metrics used for evaluation (e.g., authentication success rate, latency, security vulnerability assessment, processing overhead).

\section{Test Scenarios and Setup}

% Describe the specific scenarios designed to test the functionality, security, and performance of your solution (e.g., initial registration, re-authentication, handling of invalid credentials, identity privacy checks).

%Briefly outline the test environment configuration used for these scenarios (referencing the previous chapter but highlighting specific validation tools or configurations).

\section{Functional Validation Results}

% Present the results demonstrating that your proposed authentication and identity management mechanisms work correctly.

% Show evidence of successful device registration, authentication, and session establishment using your method (e.g., logs, packet captures, signaling flow diagrams from tests).

\section{Security Evaluation}

% Analyze the security aspects of your solution based on the test scenarios or theoretical assessment.

% Discuss how well it addresses the identified security challenges (e.g., identity concealment, resistance to specific attacks).

\section{Performance Evaluation}

% Present the results related to the performance metrics defined earlier (e.g., authentication delay, computational overhead on involved network functions).

% Compare these results against baseline scenarios or standard procedures if possible.

\section{Discussion and Analysis}

% Interpret the results presented in the previous subsections.

% Discuss how effectively your solution meets the requirements identified earlier.

% Compare your findings (functionality, security, performance) with existing solutions or the standard 3GPP methods described in the State of the Art.

% Acknowledge any unexpected results or limitations observed during validation.