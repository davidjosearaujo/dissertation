\chapter{Development And Implementation}%
\label{chapter:development-and-implementation}

\begin{introduction}
Building upon the proposed framework, this chapter describes the practical development and implementation process undertaken. In this paper, the specific tools, technologies, and configurations used to realize the solution are described, including the construction of key modules and the configuration of the experimental environment required for subsequent validation.
\end{introduction}

\section{Development Environment Setup}

% Describe the hardware, software, operating systems, and network simulators (e.g., Open5GS, UERANSIM, ns-3) or testbeds used.

% Mention specific libraries or tools employed (e.g., Python libraries for networking/crypto, Wireshark, etc.).

\section{Implementation of Proposed Authentication Logic}

% Detail how you implemented the specific EAP method or authentication flow defined in your methodology.

% Explain any code written for the device side (UE/Wi-Fi device simulator) and the network side (e.g., modifications to an EAP server, AUSF, or relevant gateway like W-AGF/TNGF).

\section{Implementation of Identity Management Mechanisms}

% Explain how you handle a device's unique identity withing the network

\section{Adaptation of Network Functions}

% Detail any modifications or specific configurations applied to simulated 5G core network functions (AMF, AUSF, UDM, potentially W-AGF/TNGF if simulating them separately).

% Explain how these functions were adapted to support the new authentication and identity schemes.

\section{System Integration and Configuration}

% Describe how the different implemented components (device, gateway, core network functions) were connected and configured to work together in your test environment.

% Include details about network configurations, interface setups, and parameter settings.

\section{Implementation Challenges}

% Briefly discuss any significant technical challenges encountered during development and how they were addressed.
