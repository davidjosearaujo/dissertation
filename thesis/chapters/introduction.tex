\chapter{Introduction}%
\label{chapter:introduction}

\begin{introduction}
A short description of the chapter.
\end{introduction}

\section{Background and Context}

\subsection{Overview of 5G Technology}

% Introduce 5G as a significant advancement in wireless technology, highlighting its core features such as higher speeds, lower latency, and support for diverse use cases (e.g., enhanced mobile broadband, massive machine-type communications, ultra-reliable low-latency communications).



\section{Problem Statement}

% - Highlight that the current 3GPP standard does not define an architecture to support Wi-Fi-only devices without USIM connecting to the 5G Core.
% - Emphasize the importance of this issue, especially in enterprise deployments where many devices lack USIM capabilities.
% - Mention the Wireless Broadband Alliance (WBA) recommendation for 3GPP to define procedures for supporting Wi-Fi-only UE with non-IMSI based identity and EAP-TLS/EAP-TTLS based authentication.

\section{Research Objectives}

% - State your primary goal: to explore and develop solutions for integrating Wi-Fi-only devices without USIM into the 5G Core infrastructure.
% - List specific objectives, such as:
%   - Investigating authentication mechanisms compatible with both 5G and Wi-Fi networks.
%   - Developing a method for device identification that works across 5G and non-3GPP networks.
%   - Proposing extensions or alternatives to existing protocols to support seamless integration.

\section{Significance of the Study}

% - Explain the importance of solving this integration challenge for the future of wireless communications.
% - Discuss potential applications and benefits in various sectors, such as enterprise networks, IoT deployments, and smart city initiatives.

\section{Thesis Structure}

% - Provide a brief overview of the subsequent chapters and their contents.