\chapter{Introduction}%
\label{chapter:introduction}

\begin{introduction}
This chapter summarizes the challenges and significance of integrating Wi-Fi-only devices into the \acl{5GC}, highlighting the limitations of current standards and the need for innovative solutions to enable seamless connectivity and authentication.
\end{introduction}

\section{Background and Context}

In recent years, \ac{5G} wireless technology as shown to have the potential to revolutionize telecommunications. It offers higher bandwidth, faster speeds, and lower delays, supporting areas such as \ac{eMBB}, \ac{mMTC}, and \ac{uRLLC}.

Additionally, it is transforming private networks, which have traditionally relied on legacy wired or wireless Ethernet.~\cite{altice-01-p3} Features like tighter security, higher reliability, and \ac{TSN} are crucial to meeting Industry 4.0 requirements for wireless connectivity. However, how can the industry bridge the gap between existing \acl{N5GC} (\ac{N5GC}) devices, which current networks rely on, and new \acp{5GC}?

\ac{5G} is not only a revolution in radio network infrastructure, but also in the core network. Now based on Service Based Architecture, using \ac{NFV} and \ac{SDN}, it's tailored to minimize cost and maximize utilization and elasticity of the infrastructure by separating the \ac{UP} functions from the \ac{CP} functions. This architecture supports various access nodes such as native \ac{NR}, LTE accesses, and non-\ac{3GPP} interworking functions that facilitate connectivity from untrusted WLANs.~\cite{23.501-p41}

As wireless networks evolve, the convergence of 5G with existing Wi-Fi infrastructures becomes increasingly critical. However, current standards established by the \ac{3GPP} do not adequately address the integration of Wi-Fi-only devices, that lack \ac{USIM} capabilities, into the \ac{5GC} network.~\cite{wba-04-2021-p59} This limitation is particularly evident in enterprise environments where many devices operate solely on Wi-Fi. To fully realize the potential of the technology, it is essential to develop solutions that enable integration of Wi-Fi-only devices into the \ac{5G} ecosystem. This includes addressing challenges related to authentication mechanisms, device identity, and overall interoperability between different network types.

\section{Problem Statement}

The current \ac{3GPP} standard lacks the hability for integrating Wi-Fi-only devices without \ac{USIM} into the \acl{5GC}, creating a significant gap in connectivity. This limitation is problematic in enterprise environments, where many devices operate only on Wi-Fi and do not possess \ac{USIM} capabilities. The \ac{WBA} has identified this issue~\cite{wba-04-2021-p59}, recommending that \ac{3GPP} should develop procedures to support Wi-Fi-only \ac{UE} using non-\acs{IMSI} based identity and authentication methods such as \ac{EAP-TLS} or \ac{EAP-TTLS}. Addressing this challenge is crucial for enabling integration of diverse device types into the \ac{5G} ecosystem.

\section{Research Objectives}

The main goal of this thesis is to explore and develop solutions for integrating Wi-Fi-only devices without \ac{USIM} into the \ac{5GC} infrastructure. To achieve this, the following objectives have been identified:

\begin{enumerate}
    \item{
        Investigate authentication mechanisms compatible with both \ac{5G} and Wi-Fi networks:
        \begin{itemize}
            \item {
                Analyze existing authentication methods such as \ac{EAP-TLS} and \ac{EAP-TTLS} for their applicability in a converged \ac{5G}-Wi-Fi environment.
            }
            \item {
                Research the current and ongoing development on 5G and Wi-Fi authentication interoperability standards.
            }
            \item {
                Explore potential modifications or extensions to these methods to ensure seamless authentication across different network types.
            }
        \end{itemize}
    }
    \item{
        Develop a method for managing device identity that works across \ac{5G} and non-\ac{3GPP} networks:
        \begin{itemize}
            \item Investigate the possibility of designing an extended \ac{NAI} or alternative identifier that can accommodate Wi-Fi-only devices while maintaining compatibility with the \ac{5G} infrastructure.
            \item Investigate the possibility of generating a pseudo-\ac{SUCI} and pseudo-\ac{SUPI} for \ac{N5GC} devices that follows the \ac{NAI} format and serves a similar function in the authentication flow.
        \end{itemize}
    }
    \item{
        Propose extensions or alternatives to existing protocols:
        \begin{itemize}
            \item Investigate the possibility for mapping existing \ac{N5GC} device identifiers (e.g., \ac{MAC} addresses) to a format compatible with the \ac{5G} authentication framework.
            \item Explore the potential for enhancing or creating new \ac{EAP} methods specifically designed for \ac{N5GC} devices in a \ac{5G} context.
        \end{itemize}
    }
\end{enumerate}

\section{Thesis Structure}

This document explores the challenge of integrating Wi-Fi-only devices into the \ac{5GC}. We begin by examining the current landscape of \ac{5G} and Wi-Fi integration, focusing on authentication mechanisms and their limitations. Building on this foundation, a framework to address these challenges is proposed, detailing the targeted approach for security. The solution is then put to the test, presenting experimental results and comparing them with existing methods. Finally, a reflection on achieved contributions is given, acknowledging the boundaries of the work, and suggest possible avenues for future research. Through this journey, the aim is to meaningfully contribute to the ongoing convergence of \ac{5G} and Wi-Fi technologies.