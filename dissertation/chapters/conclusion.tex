\chapter{Conclusion}
\label{chapter:conclusion}

\begin{introduction}
Concluding the dissertation, this chapter synthesizes the research conducted, summarizing the key findings and contributions toward integrating Wi-Fi-only devices in \ac{5G} environments. Re-visits the problem statement and research objectives, discusses the implications of the results, acknowledges the limitations encountered during the study, and proposes potential avenues for future research and development in this domain.
\end{introduction}

\section{Comparison with Standard \ac{3GPP} Methods and State of the Art}

Standard \ac{3GPP} mechanisms for non-\ac{3GPP} access typically involve the \ac{UE} itself (or a function like \ac{N3IWF}/\ac{TNGF}) handling the interface to the \ac{5GC}. For devices behind an \ac{5G-RG} that are not \ac{5G}-capable (\ac{NAUN3}), \ac{3GPP} TS 23.316 discusses "Connectivity Group IDs"~\cite{23.316-p27} where groups of devices on a \ac{LAN} segment can be mapped to a \ac{PDU} session established by the \ac{5G-RG}. More recent Release 19 additions~\cite{23.316-p29}~\cite{23.316-p95}~\cite{23.501-p564} introduce the "Non-\ac{3GPP} Device Identifier" to allow for differentiated \ac{QoS} for individual devices within a \ac{PDU} session.

This project's solution aligns with the trend of providing more granular management for devices behind an \ac{5G-RG}. However, it offers a distinct approach:

\begin{itemize}
    \item \textbf{Authentication:} Standard \ac{3GPP} methods~\cite{23.501-p564} state that such non-\ac{3GPP} devices are not directly authenticated by the \ac{5GC}. This project implements a robust \ac{EAP-TLS} local authentication step as a prerequisite for any \ac{5G} resource allocation, a crucial aspect not explicitly detailed for \ac{NAUN3} devices in the standard flows for \ac{QoS} differentiation.

    \item \textbf{Identity Management:} While \ac{3GPP} R19 uses "Non-\ac{3GPP} Device Identifiers" mainly for \ac{QoS} differentiation within a \ac{PDU} session, this project uses the dedicated \ac{PDU} session itself as the primary proxy identity for the \ac{NAUN3} device in the \ac{5G-RG}. This provides a stronger isolation boundary and a direct handle for per-device policy and \ac{IP} management by the \ac{5G-RG}. The \ac{3GPP} approach~\cite{23.316-p95} can lead to a \ac{PDU} Session Modification or Establishment for differentiated \ac{QoS}, which is functionally similar in outcome for a single device but our approach makes this one-to-one mapping fundamental.
\end{itemize}

The implemented solution can be seen as a specific instantiation of how a \ac{5G-RG} could manage \ac{NAUN3} devices, extending the concept of Connectivity Groups to a per-device granularity and integrating a necessary local authentication layer.

\section{Limitations and Unexpected Results}

\begin{itemize}
    \item \textbf{Onboarding Delay:} %The measured onboarding delay of approximately 135 seconds (naun301_auth_and_connection_delay.txt) is significant. This duration includes the full \ac{EAP-TLS} handshake, multiple \ac{RADIUS} message exchanges (potentially including retransmissions if the backhaul \ac{PDU} session \ac{IP} is not immediately used by \ texttt{hostapd}, as observed in RADIUS_traffic_via_backhaul.jpg and \ texttt{hostapd}.txt), the interceptor's \ac{PDU} session establishment process (which involves CLI command execution and polling), and local DHCP. While functional, this delay would be a concern for user experience in a real-world deployment and suggests areas for optimization, such as more efficient \ac{PDU} session triggering (e.g., via an API instead of CLI polling if \ac{UE}RANSIM/modem supported it) or faster \ac{RADIUS} resolution.

    \item Performance Metrics: %While iperf3 tests confirmed data plane connectivity and isolation, this validation did not focus on rigorous performance benchmarking (e.g., maximum throughput, detailed latency analysis under load, impact of the \ texttt{iptables} rules on the \ac{RG}-\ac{RG}'s performance). Such analysis would require a more specialized test setup.

    \item \textbf{Scalability:} The current validation involved a small number of \ac{NAUN3} devices. The scalability of the interceptor application, particularly its management of \ac{PDU} sessions, routing rules, and state for a large number of simultaneously connected \ac{NAUN3} devices, was not extensively tested. The reliance on \texttt{nr-cli} and system commands for each device could become a bottleneck.

    \item Security of Orchestration: While \ac{EAP-TLS} provides strong authentication and \ac{RADIUS} can be secured (e.g., over \ac{IPsec}, though not explicitly implemented in the backhaul \ac{PDU} beyond \ac{GTP-U} tunneling), the security of the interceptor application itself and its control interfaces (e.g., \texttt{hostapd} control socket) would require further hardening in a production environment.

    \item \textbf{Complexity of Traffic Mapping:} The dynamic creation and deletion of policy-based routing rules and \texttt{iptables} entries, while effective, adds complexity to the \ac{5G-RG}. Managing these rules for many devices without conflicts or performance degradation requires careful implementation.
\end{itemize}

In summary, the validation demonstrates that the proposed framework successfully addresses the core requirements for authenticating \ac{NAUN3} devices and integrating them into a \ac{5G} network using a novel proxy identity mechanism. The results highlight the functional correctness of the solution while also pointing to areas such as onboarding delay and scalability as considerations for future refinement or real-world deployment.

\section{Envisioned Enhancement}

\subsection{User-Specific \ac{\ac{QoS}} Policies}

While the current implementation effectively conceals \ac{NAUN3} identities from the \ac{\ac{RG}C}, a future enhancement could involve a secure communication channel between the \ac{RADIUS} server and the \ac{5GC} (e.g., \ac{PCF}/\ac{UDM}). Upon successful \ac{EAP-TLS} authentication, the \ac{RADIUS} server could inform the \ac{5GC} about the authenticated \ac{EAP} identity (which could be linked to a broader user or device profile known to the operator), the \ac{MAC} address of the \ac{NAUN3} device, and the \ac{5G-RG}'s identity (e.g., its \ac{SUPI} or the \ac{IP} of its backhaul \ac{PDU} session from which the \ac{RADIUS} request was relayed). To apply user-specific \ac{QoS}, the \ac{5GC} would then need to query the \ac{5G-RG} (which maintains the \ac{MAC}-to-\ac{PDU}-session mapping) to identify which specific \ac{PDU} session (established under the \ac{5G-RG}'s \ac{SUPI}) corresponds to the target \ac{NAUN3} device's \ac{MAC} address. Once this correlation is made, the \ac{5GC} (specifically the \ac{PCF}) could apply user-specific \ac{QoS} policies to this now-identified \ac{PDU} session, even if the \ac{NAUN3} device itself doesn't have a traditional \ac{IMSI}-based subscription. This would allow for a richer, policy-driven service differentiation based on the authenticated local identity, bridging the local authentication domain with the \ac{5GC}'s policy framework without exposing \ac{NAUN3} \ac{MAC} addresses directly during \ac{PDU} session establishment by the \ac{5G-RG}.