\chapter{Conclusion}%
\label{chapter:conclusion}

\begin{introduction}
Concluding the dissertation, this chapter synthesizes the research conducted, summarizing the key findings and contributions toward integrating Wi-Fi-only devices in \ac{5G} environments. Re-visits the problem statement and research objectives, discusses the implications of the results, acknowledges the limitations encountered during the study, and proposes potential avenues for future research and development in this domain.
\end{introduction}

\section{Summary of Research and Key Findings}

This research confronted the intricate problem of integrating Wi-Fi-only devices, which fit under \ac{NAUN3} devices as per \ac{3GPP} definition, into the \ac{5G} System. These devices, lacking native \ac{5G} credentials such as a \ac{USIM}, are inherently unable to directly authenticate with or be managed by the \ac{5GC} through standard \ac{5G} procedures. The primary objectives were therefore to design a robust and secure local authentication mechanism for these devices, to propose an innovative identity management solution enabling the \ac{5GC} to handle their traffic without necessitating modifications to the devices or the core network \acp{NF}, and to validate this comprehensive framework within a realistic, albeit simulated, \ac{5G} environment.

The proposed solution is designed around the usage of a \ac{5G-RG}, which functions as an intelligent mediation point. The framework first mandates a local \ac{EAP-TLS} authentication process for each \ac{NAUN3} device, where the \ac{5G-RG} acts as an \ac{EAP} authenticator (or relay), forwarding authentication dialogues to an external, operator-controlled \ac{RADIUS} server. Upon successful local authentication, the cornerstone of the solution is activated: the \ac{5G-RG}, orchestrated by a custom software component (the \texttt{interceptor}), requests a dedicated \ac{PDU} Session from the \ac{5GC} specifically for that authenticated \ac{NAUN3} device. This \ac{PDU} Session, established on a designated \texttt{clients} \ac{DNN}, effectively serves as a dynamic "proxy identity" for the \ac{NAUN3} device within the \ac{5G} system. The \texttt{interceptor} is crucial in linking the outcome of the local authentication to the subsequent \ac{5G} \ac{PDU} session management and the dynamic configuration of traffic routing rules. To maintain a clear separation of traffic, a distinct \texttt{backhaul} \ac{DNN} is utilized for the \ac{5G-RG}'s own operational communications, such as the \ac{RADIUS} messages exchanged with the \ac{EAP} server.

The validation phase, conducted through a series of experiments in a simulated multi-\ac{VM} environment, confirmed the framework's functional success. Key achievements include the consistent and secure \ac{EAP-TLS} authentication of \ac{NAUN3} devices, as evidenced by detailed logs from all involved components (\texttt{wpa\_supplicant}, 	exttt{hostapd}, FreeRADIUS, and the interceptor). Following authentication, the dynamic establishment of unique \ac{PDU} sessions on the \texttt{clients} \ac{DNN} for each \ac{NAUN3} device, along with the creation of corresponding \texttt{uesimtunX} network interfaces on the \ac{5G-RG}, was achieved. Both local (via \ac{DHCP} managed by \texttt{dnsmasq}) and \ac{5GC}-assigned \ac{IP} addresses were correctly allocated. End-to-end data connectivity was verified through \texttt{ping -R} tests and \texttt{iperf3} traffic generation, which also confirmed correct \ac{NAT} operation via the per-device \ac{PDU} session \acp{IP}. Traffic isolation between multiple \ac{NAUN3} devices, each utilizing its unique \ac{PDU} session, was successfully demonstrated, supported by the dynamic \texttt{iptables} and policy routing rules. The system also effectively managed the lifecycle of these resources, cleaning up \ac{PDU} sessions and associated configurations upon simulated device disconnections. Furthermore, control plane traffic (\ac{RADIUS}/\ac{EAP}) was shown to be correctly segregated onto the backhaul \ac{DNN}, and the \ac{NAUN3} device's local identity remained concealed from the \ac{5GC} \acp{NF}. Repeated tests indicated an average onboarding delay of approximately 27 seconds.

\section{Contributions of the Dissertation}

This dissertation offers several significant contributions to the challenge of integrating diverse, non-\ac{5G}-native devices into contemporary \ac{5G} networks.

The primary contribution is the design and proof-of-concept implementation of a practical, gateway-centric framework that enables Wi-Fi-only/\ac{NAUN3} devices, which lack standard \ac{5G} credentials, to securely access \ac{5G} network services. This solution is particularly relevant for scenarios where modifying end-user devices or the \ac{5GC} is not feasible.

A key innovation is the novel application of per-device \ac{PDU} Sessions as a dynamic proxy identity mechanism. Managed by the \ac{5G-RG}, this approach allows the \ac{5GC} to handle traffic for individual \ac{NAUN3} devices without requiring these devices to have a \ac{SUPI} or undergo direct \ac{5GC} authentication. Each \ac{PDU} session acts as a distinct, manageable network presence for the corresponding \ac{NAUN3} device.

The research also demonstrates how local, strong authentication (\ac{EAP-TLS}) can be tightly coupled with \ac{5G} session management procedures at the network edge (\ac{5G-RG}). This ensures that only verified devices are granted access to \ac{5G} resources via these proxy \ac{PDU} sessions, addressing a critical security consideration.

The development of a working proof-of-concept in a simulated \ac{5G} environment using open-source tools (Open5GS, UERANSIM) and custom orchestration logic also validates the architectural design. This implementation confirms that the proposed framework can operate with minimal impact on standard \ac{5G} \acp{NF} (requiring only configuration) and on the \ac{NAUN3} devices themselves (requiring only standard \ac{EAP} supplicant capabilities).

Collectively, this work addresses a significant identity and authentication gap for a specific but increasingly common class of devices, offering a pathway for their managed integration into the \ac{5G} ecosystem, thereby enhancing the versatility and reach of \ac{5G} services.

\subsection{Comparison with Standard \acs{3GPP} Methods and State of the Art}

Standard \ac{3GPP} mechanisms for non-\ac{3GPP} access typically involve the \ac{UE} itself (or a function like \ac{N3IWF}/\ac{TNGF}) handling the interface to the \ac{5GC}. For devices behind an \ac{5G-RG} that are not \ac{5G}-capable (\ac{NAUN3}), \ac{3GPP} TS 23.316 discusses "Connectivity Group IDs"~\cite{23.316-p27} where groups of devices on a \ac{LAN} segment can be mapped to a \ac{PDU} session established by the \ac{5G-RG}. More recent Release 19 additions~\cite{23.316-p29}~\cite{23.316-p95}~\cite{23.501-p564} introduce the "Non-\ac{3GPP} Device Identifier" to allow for differentiated \ac{QoS} for individual devices within a \ac{PDU} session.

This project's solution aligns with the trend of providing more granular management for devices behind an \ac{5G-RG}. However, it offers a distinct approach:

\begin{itemize}
    \item \textbf{Explicit Local Authentication as a Prerequisite:} Standard \ac{3GPP} methods~\cite{23.501-p564} state that such non-\ac{3GPP} devices are not directly authenticated by the \ac{5GC}. This project implements a robust \ac{EAP-TLS} local authentication step as a prerequisite for any \ac{5G} resource allocation, a crucial aspect not explicitly detailed for \ac{NAUN3} devices in the standard flows for \ac{QoS} differentiation.

    \item \textbf{\ac{PDU} Session as Proxy Identity vs. \ac{QoS} Tag:} While \ac{3GPP} R19 uses "Non-\ac{3GPP} Device Identifiers" mainly for \ac{QoS} differentiation within a \ac{PDU} session, this project uses the dedicated \ac{PDU} session itself as the primary proxy identity for the \ac{NAUN3} device in the \ac{5G-RG}. This provides a stronger isolation boundary and a direct handle for per-device policy and \ac{IP} management by the \ac{5G-RG}. The \ac{3GPP} approach~\cite{23.316-p95} can lead to a \ac{PDU} Session Modification or Establishment for differentiated \ac{QoS}, which is functionally similar in outcome for a single device but our approach makes this one-to-one mapping fundamental.
\end{itemize}

The implemented solution can be seen as a specific instantiation of how a \ac{5G-RG} could manage \ac{NAUN3} devices, extending the concept of Connectivity Groups to a per-device granularity and integrating a necessary local authentication layer.

\section{Discussion of Limitations}

While the validation phase confirmed the functional viability of the proposed framework, several limitations were identified, warranting consideration for future work and potential real-world deployments:

A notable limitation is the onboarding delay. Systematic testing revealed an average end-to-end delay of approximately 33.1634193 seconds (and 5.0078050 seconds of standard diviation) from the initiation of the \texttt{wpa\_supplicant} on the \ac{NAUN3} device to its acquisition of a local \ac{IP} address. This comprehensive duration includes the \ac{EAP-TLS} handshake, all \ac{RADIUS} communications, the \ac{PDU} session establishment orchestrated by the interceptor (which relies on \ac{CLI} command execution and polling for UERANSIM), and the final local \ac{DHCP} lease acquisition. While a significant improvement from some initial outlier measurements, and functional for a proof-of-concept, this delay could impact user experience in time-sensitive applications and represents a key area for optimization.

The data plane performance and scalability of the solution were confirmed at a functional level but not rigorously benchmarked. While \texttt{iperf3} tests demonstrated successful data transfer and isolation for a small number of concurrent devices, with observed bitrates suitable for many common applications, the study did not extend to stress testing under heavy load or with a large number of \ac{NAUN3} devices. The reliance of the \texttt{interceptor} application on executing system commands (\texttt{nr-cli}, \texttt{iptables}, \texttt{ip route}/\texttt{rule}) for each device's lifecycle events could potentially become a performance bottleneck on the \ac{5G-RG} at higher scales. A detailed analysis of \ac{CPU}/memory impact on the \ac{5G-RG} under such conditions was also outside the scope of this work.

The use of \ac{NAT} for traffic mapping from \ac{NAUN3} devices to their respective \ac{PDU} Sessions, while effective for enabling outbound connectivity and providing a unique external \ac{IP} presence per device, inherently introduces limitations typically associated with \ac{NAT}. This primarily restricts the ease of initiating connections to the \ac{NAUN3} devices from external parties in the data network, as these devices are not directly addressable by their local \acp{IP} from the outside. Similarly, direct peer-to-peer communication between two \ac{NAUN3} devices, each behind its own \ac{NAT} on the \ac{5G-RG} (if their traffic is routed externally and back), would require \ac{NAT} traversal techniques or application-level relaying, which were not explored in this work.

Regarding security, while \ac{EAP-TLS} provides strong local authentication for the \ac{NAUN3} devices and traffic segregation was demonstrated, the custom \texttt{interceptor} application itself and its control interfaces (e.g., the \texttt{hostapd} control socket) would necessitate further security hardening and thorough vulnerability assessment before any consideration for production deployment.

The complexity of dynamic traffic management on the \ac{5G-RG} is another consideration. The \texttt{interceptor}'s routing handler module successfully managed \texttt{iptables} and policy-based routing rules for the tested scenarios. However, ensuring conflict-free, secure, and performant rule management for a very large and highly dynamic set of devices would introduce significant operational complexity.

Finally, the attempt to integrate a physical \ac{5G} RedCap modem (Quectel RG500Q-GL) highlighted substantial physical hardware integration challenges. Issues related to proprietary drivers, kernel version dependencies, and a lack of comprehensive public documentation for advanced multi-\ac{PDU} session features (like \ac{QMAP}) made it difficult to replicate the fine-grained \ac{PDU} session control achieved in the UERANSIM-based simulated environment. This underscores the potential gap between simulated proofs-of-concept and the practical deployment on diverse physical \ac{RG} platforms, which would likely require considerable modem-specific adaptation and driver-level work.

\section{Envisioned Enhancement}

The findings and limitations of this research open several avenues for future investigation and development to enhance the proposed framework:

A primary focus should be on optimization of the onboarding delay. This could involve exploring more efficient mechanisms for \ac{PDU} session control by the \texttt{interceptor}, such as direct \ac{API}-based interactions with the \ac{5G-RG}'s \ac{UE} stack or modem if supported by future UERANSIM versions or different hardware platforms, thereby avoiding \ac{CLI} parsing and polling. Optimizing \ac{RADIUS} server response times and the \ac{EAP-TLS} exchange, or even investigating the feasibility of pre-establishing a pool of \texttt{clients} \ac{DNN} \ac{PDU} sessions that can be rapidly assigned to newly authenticated devices, could also yield significant improvements.

Comprehensive performance and scalability analysis is essential. Future work should involve rigorous testing with a substantially larger number of concurrent \ac{NAUN3} devices under various traffic load conditions. This would help identify potential bottlenecks in the \ac{5G-RG} (\ac{CPU}, memory), the \texttt{interceptor} application, or the \ac{5GC}, and quantify the framework's scalability limits. Exploring more performant traffic mapping mechanisms, such as \ac{eBPF}-based solutions as an alternative to \texttt{iptables} for large-scale deployments, could also be beneficial.

Enhanced security hardening of the \ac{5G-RG} environment and the \texttt{interceptor} application is crucial. This includes securing the control interfaces used by the \texttt{interceptor}, implementing robust input validation, and potentially employing secure tunneling mechanisms like \ac{IPSec} for \ac{RADIUS} traffic over the backhaul \ac{PDU} session, especially if it traverses untrusted network segments.

Further research could explore deeper integration with advanced \ac{5G} features. The per-device \ac{PDU} session model provides a strong foundation for applying granular policies. Future work could investigate how this could be leveraged for fine-grained network slicing per \ac{NAUN3} device or dynamic \ac{QoS} adjustments through integration with the \ac{5GC}'s Policy Control Function (\ac{PCF}). This could build upon the envisioned enhancement where the \ac{RADIUS} server and \ac{5G-RG} securely communicate authenticated local identity information (\ac{EAP} identity, \ac{MAC} address) to the \ac{PCF}/\ac{UDM}, allowing the \ac{PCF} to apply user-specific profiles to the correct proxy \ac{PDU} session.

Expanding support for other authentication methods beyond \ac{EAP-TLS} could increase the framework's applicability. This might include exploring \ac{MAC}-based authentication for very simple \ac{IoT} devices (with careful consideration of the associated security implications) or other \ac{EAP} types suitable for different device capabilities and security requirements.

Addressing the limitations of NAT for inbound connections could also be explored, perhaps through integration with \ac{UPF} capabilities for port forwarding or specific application-level gateways if required for certain \ac{NAUN3} device services. One potential avenue to mitigate \ac{NAT}-related issues and improve inbound addressability for \ac{NAUN3} devices could be the exploration of \textit{Framed Routing}. In this scenario, the \ac{RADIUS} server, upon successful authentication, could provide \textit{Framed-Route} attributes to the \ac{5G-RG}. These attributes would suggest static routes to be installed on the \ac{5G-RG}, potentially assigning a routable \ac{IP} address or a small subnet directly to the \ac{NAUN3} device's \ac{PDU} session. This would require the \ac{5G-RG} to install these routes and ensure that the \ac{5GC} (specifically \ac{SMF}/\ac{UPF}) is also aware of how to route traffic for these Framed-\ac{IP}-Addresses towards the correct \ac{5G-RG} and subsequently to the \ac{PDU} session tunnel associated with the \ac{NAUN3} device. Such an approach could enable more direct inbound connectivity and facilitate peer-to-peer communication, though it would necessitate careful coordination of \ac{IP} address management and routing policies between the \ac{RADIUS} infrastructure, the \ac{5G-RG}, and the \ac{5GC}.

Addressing the challenges of physical modem integration remains a significant area. Continued efforts to work with diverse physical \ac{5G} modems and \ac{RG} platforms, including deeper investigation into modem-specific \acp{API}, \ac{QMAP} functionalities, and driver development or adaptation, would be necessary for real-world deployment. Findings from such work could also inform potential contributions to standardization efforts if gaps in managing \ac{NAUN3}-type devices via \acp{RG} persist.

Finally, investigating mobility scenarios, such as when the \ac{5G-RG} itself is mobile or when \ac{NAUN3} devices roam between different local access points (potentially managed by different \acp{5G-RG}, would be a valuable extension to assess the framework's robustness and adaptability in more dynamic network conditions.

\subsection{User-Specific \acs{QoS} Policies}

During the development of the current solution, another was envisioned for a possible future iteration. While the current implementation effectively conceals \ac{NAUN3} identities from the \ac{5GC}, a future enhancement could involve a secure communication channel between the \ac{RADIUS} server and the \ac{5GC} (e.g., \ac{PCF}/\ac{UDM}). Upon successful \ac{EAP-TLS} authentication, the \ac{RADIUS} server could inform the \ac{5GC} about the authenticated \ac{EAP} identity (which could be linked to a broader user or device profile known to the operator), the \ac{MAC} address of the \ac{NAUN3} device, and the \ac{5G-RG}'s identity (e.g., its \ac{SUPI} or the \ac{IP} of its \texttt{backhaul} \ac{PDU} session from which the \ac{RADIUS} request was relayed). To apply user-specific \ac{QoS}, the \ac{5GC} would then need to query the \ac{5G-RG} (which maintains the \ac{MAC}-to-\ac{PDU}-session mapping) to identify which specific \ac{PDU} session (established under the \ac{5G-RG}'s \ac{SUPI}) corresponds to the target \ac{NAUN3} device's \ac{MAC} address. Once this correlation is made, the \ac{5GC} (specifically the \ac{PCF}) could apply user-specific \ac{QoS} policies to this now-identified \ac{PDU} session, even if the \ac{NAUN3} device itself doesn't have a traditional \ac{IMSI}-based subscription. This would allow for a richer, policy-driven service differentiation based on the authenticated local identity, bridging the local authentication domain with the \ac{5GC}'s policy framework without exposing \ac{NAUN3} \ac{MAC} addresses directly during \ac{PDU} session establishment by the \ac{5G-RG}.

\section{Final Concluding Remarks}

This dissertation has successfully designed, implemented, and validated a novel gateway-centric framework for the secure integration of Wi-Fi-only/\ac{NAUN3} devices into \ac{5G} networks. By innovatively employing local \ac{EAP-TLS} authentication as a prerequisite for the dynamic establishment of per-device \ac{PDU} sessions—which act as proxy identities within the \ac{5GC}, this research addresses a critical gap in current \ac{5G} architectures concerning the authentication and individualized management of devices lacking native \ac{5G} credentials. The validation results robustly demonstrate the functional correctness of this approach, including secure device onboarding, unique \ac{PDU} session allocation, effective traffic isolation and mapping, and proper resource lifecycle management, all achieved with minimal impact on the standard \ac{5GC} and the \ac{NAUN3} end devices.

The primary takeaway from this work is that a \ac{5G-RG}, augmented with custom orchestration logic, can serve as an effective mediation point to bridge the local network domain of unauthenticated devices with the credential-based \ac{5G} system. This provides a practical pathway for extending \ac{5G} services to a broader range of devices. While acknowledging the identified limitations, such as the onboarding delay and the need for further scalability and performance testing, this research lays a solid foundation for future enhancements. The proposed solution not only offers a tangible approach to a current integration challenge but also aligns with the evolving \ac{3GPP} vision of more granular service control for devices in converged network environments. Ultimately, this work contributes to the broader goal of creating a more inclusive, flexible, and truly ubiquitous \ac{5G} ecosystem.